\documentclass{article}
\usepackage{amsmath}

\title{
    UZ summary \\
    \large Summary of the Machine Perception course lectures at FRI
}

\begin{document}
\maketitle
\newpage
\tableofcontents
\newpage

\section{Image processing}

Processing steps:
\begin{enumerate}
    \item Convert gray image to binary image \textbf{(thresholding)}
    \item Clean binary image \textbf{(morphologic filtering)}
    \item Extract individual regions \textbf{(connected components)}
\end{enumerate}

    \subsection{Thresholding}
    Otsu's algorithm:
    \begin{enumerate}
        \item Separate the pixels into two groups by intensity threshold $T$
        \item For each group get an average intensity and calculate $\sigma^{2}_{between}$
        \item Select the $T$ that maximizes the variance: \\
                  $T^* = argmax_T[ \sigma^{2}_{between}(T) ]$
    \end{enumerate}

    \subsection{Cleaning the image}
    
        \subsubsection{Fitting and hitting}
        \textbf{Fitting:} all ``1'' pixels in the SE (structuring element) cover all ``1'' pixels in the image \\ 
        \textbf{Hitting:} at least one ``1'' pixels in the SE covers a ``1'' pixel in the image

        \subsubsection{Erosion}
        \begin{itemize}
            \item Reduces the size of structures
            \item Removes bridges, branches, noise
            \item $g(x, y) = 
            \begin{cases}
                1 \text{ if } s \text{ fits } f \\
                0 \text{ otherwise }
            \end{cases}$
        \end{itemize}

        \subsubsection{Dilation}
        \begin{itemize}
            \item Increases the size of structures
            \item Fills holes in regions
            \item $g(x, y) =
            \begin{cases}
                1 \text{ if } s \text{ hits } f \\
                0 \text{ otherwise }
            \end{cases}$
        \end{itemize}

        \subsubsection{Opening}
        \begin{itemize}
            \item \textbf{Erosion} followed by \textbf{dilation}
            \item Removes small objects, preserves rough shape
            \item Filters out structures depending on the size and shape of the structuring element
        \end{itemize}

        \subsubsection{Closing}
        \begin{itemize}
            \item \textbf{Dilation} followed by \textbf{erosion}
            \item Fills holes, preserves rough shape
        \end{itemize}

\end{document}
