\documentclass{article}
\usepackage{amsmath}

\title{
    UZ summary \\
    \large Summary of the Machine Perception course lectures at FRI
}

\begin{document}
\maketitle
\newpage
\tableofcontents
\newpage

\section{Image processing}

Processing steps:
\begin{enumerate}
    \item Convert gray image to binary image \textbf{(thresholding)}
    \item Clean binary image \textbf{(morphologic filtering)}
    \item Extract individual regions \textbf{(connected components)}
\end{enumerate}

    \subsection{Thresholding}
    Otsu's algorithm:
    \begin{enumerate}
        \item Separate the pixels into two groups by intensity threshold $T$
        \item For each group get an average intensity and calculate $\sigma^{2}_{between}$
        \item Select the $T$ that maximizes the variance: \\
                  $T^* = argmax_T[ \sigma^{2}_{between}(T) ]$
    \end{enumerate}

    \subsection{Cleaning the image}
    
        \subsubsection{Fitting and hitting}
        \textbf{Fitting:} all ``1'' pixels in the SE (structuring element) cover all ``1'' pixels in the image \\ 
        \textbf{Hitting:} at least one ``1'' pixels in the SE covers a ``1'' pixel in the image

        \subsubsection{Erosion}
        \begin{itemize}
            \item Reduces the size of structures
            \item Removes bridges, branches, noise
            \item $g(x, y) = 
            \begin{cases}
                1 \text{ if } s \text{ fits } f \\
                0 \text{ otherwise }
            \end{cases}$
        \end{itemize}

        \subsubsection{Dilation}
        \begin{itemize}
            \item Increases the size of structures
            \item Fills holes in regions
            \item $g(x, y) =
            \begin{cases}
                1 \text{ if } s \text{ hits } f \\
                0 \text{ otherwise }
            \end{cases}$
        \end{itemize}

        \subsubsection{Opening}
        \begin{itemize}
            \item \textbf{Erosion} followed by \textbf{dilation}
            \item Removes small objects, preserves rough shape
            \item Filters out structures depending on the size and shape of the structuring element
        \end{itemize}

        \subsubsection{Closing}
        \begin{itemize}
            \item \textbf{Dilation} followed by \textbf{erosion}
            \item Fills holes, preserves rough shape
        \end{itemize}

    \subsection{Labelling regions}

        \subsubsection{Sequential connected components}
        \begin{itemize}
            \item Process image from left to right, from top to bottom:
            \begin{enumerate}
                \item If the current pixel value is 1
                \begin{enumerate}
                    \item If only one neighbor \textbf{(left or top)} is 1, copy its label
                    \item If both neighbors are 1 and have the same label, copy the label
                    \item If they have different labels:
                    \begin{itemize}
                        \item Copy label from the left
                        \item Update the table of equivalent labels (remember that the label of the left neighbor represents the same connected component as the label of the top neighbor)
                    \end{itemize}
                    \item Otherwise form a new label
                \end{enumerate}
            \end{enumerate}
            \item Relabel with the smallest equivalent labels
        \end{itemize}

    \subsection{Color}

    \begin{itemize}
        \item Additive models
        \item Subtractive models
    \end{itemize}

        \subsubsection{Color spaces}
        \begin{itemize}
            \item Role: unique color specification
            \item A color space is defined by the choice of primary colors (primaries)
        \end{itemize}
                \subsubsection{Non-uniform color spaces}
                \begin{itemize}
                    \item If two colors are close to each other (by Euclidean distance), it \textbf{doesn't mean that they are similar perceptually}
                    \item CIE XYZ --- linear
                    \item RGB --- linear
                    \item HSV --- nonlinear
                \end{itemize}

                \subsubsection{Uniform color spaces}
                \begin{itemize}
                    \item CIE Lab
                    \item CIE u'v'
                \end{itemize}

        \subsection{Color description by using histograms}
        \begin{itemize}
            \item Histograms record the frequency of intensity levels \\
                  $h(i) = \text{ the number of pixels in } I \text{ with the intensity value } i$
            \item Color histogram is a robust representation of images (robust to changes in translation, scale and partial occlusion)
        \end{itemize}

        \subsection{Filtering}
        \begin{itemize}
            \item Noise reduction and image restoration
            \item Structure extraction/enhancement
        \end{itemize}

            \subsubsection{Types of noise}
            \begin{itemize}
                \item Gaussian noise
                \item Salt and pepper noise
                \item Impulse noise
            \end{itemize}

            When chaining \textbf{linear} filters, the order \textbf{doesn't matter}.
            When chaining \textbf{nonlinear} filters (e.g.\ the median filter), the order \textbf{does matter}.

        \subsection{Linear filtering as template atching}
        \begin{itemize}
            \item If we correlate the image with a template, we get a map of the template's similarity to the image
            \item The max value of the map is the location of the template in the image
            \item To account for the scale dissimilarities, we can start with the original version of our image, correlate with the template, scale the image \textbf{down} (subsample) and repeat the process --- \textbf{image pyramid}
            \item \textbf{Smoothing} the image before subsampling removes the features that couldn't be reconstructed in the subsampled image because of \textbf{antialiasing}
        \end{itemize}
        
\end{document}
